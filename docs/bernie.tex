\PassOptionsToPackage{unicode=true}{hyperref} % options for packages loaded elsewhere
\PassOptionsToPackage{hyphens}{url}
%
\documentclass[]{book}
\usepackage{lmodern}
\usepackage{amssymb,amsmath}
\usepackage{ifxetex,ifluatex}
\usepackage{fixltx2e} % provides \textsubscript
\ifnum 0\ifxetex 1\fi\ifluatex 1\fi=0 % if pdftex
  \usepackage[T1]{fontenc}
  \usepackage[utf8]{inputenc}
  \usepackage{textcomp} % provides euro and other symbols
\else % if luatex or xelatex
  \usepackage{unicode-math}
  \defaultfontfeatures{Ligatures=TeX,Scale=MatchLowercase}
\fi
% use upquote if available, for straight quotes in verbatim environments
\IfFileExists{upquote.sty}{\usepackage{upquote}}{}
% use microtype if available
\IfFileExists{microtype.sty}{%
\usepackage[]{microtype}
\UseMicrotypeSet[protrusion]{basicmath} % disable protrusion for tt fonts
}{}
\IfFileExists{parskip.sty}{%
\usepackage{parskip}
}{% else
\setlength{\parindent}{0pt}
\setlength{\parskip}{6pt plus 2pt minus 1pt}
}
\usepackage{hyperref}
\hypersetup{
            pdftitle={My Bernie Story},
            pdfauthor={Rory Nolan},
            pdfborder={0 0 0},
            breaklinks=true}
\urlstyle{same}  % don't use monospace font for urls
\usepackage{longtable,booktabs}
% Fix footnotes in tables (requires footnote package)
\IfFileExists{footnote.sty}{\usepackage{footnote}\makesavenoteenv{longtable}}{}
\usepackage{graphicx,grffile}
\makeatletter
\def\maxwidth{\ifdim\Gin@nat@width>\linewidth\linewidth\else\Gin@nat@width\fi}
\def\maxheight{\ifdim\Gin@nat@height>\textheight\textheight\else\Gin@nat@height\fi}
\makeatother
% Scale images if necessary, so that they will not overflow the page
% margins by default, and it is still possible to overwrite the defaults
% using explicit options in \includegraphics[width, height, ...]{}
\setkeys{Gin}{width=\maxwidth,height=\maxheight,keepaspectratio}
\setlength{\emergencystretch}{3em}  % prevent overfull lines
\providecommand{\tightlist}{%
  \setlength{\itemsep}{0pt}\setlength{\parskip}{0pt}}
\setcounter{secnumdepth}{5}
% Redefines (sub)paragraphs to behave more like sections
\ifx\paragraph\undefined\else
\let\oldparagraph\paragraph
\renewcommand{\paragraph}[1]{\oldparagraph{#1}\mbox{}}
\fi
\ifx\subparagraph\undefined\else
\let\oldsubparagraph\subparagraph
\renewcommand{\subparagraph}[1]{\oldsubparagraph{#1}\mbox{}}
\fi

% set default figure placement to htbp
\makeatletter
\def\fps@figure{htbp}
\makeatother

\usepackage{booktabs}
\usepackage[]{natbib}
\bibliographystyle{apalike}

\title{My Bernie Story}
\author{Rory Nolan}
\date{2020-08-02}

\begin{document}
\maketitle

{
\setcounter{tocdepth}{1}
\tableofcontents
}
\hypertarget{preface}{%
\chapter*{Preface}\label{preface}}
\addcontentsline{toc}{chapter}{Preface}

\hypertarget{ways-to-read}{%
\section*{Ways to read}\label{ways-to-read}}
\addcontentsline{toc}{section}{Ways to read}

This thesis may be read on the web at \url{https://rorynolan.github.io/bernie/}. If you are reading this on the web now but would like a PDF or EPUB version, click on the download symbol at the top left of the page (to the right of the \textbf{A}).

\hypertarget{intro}{%
\chapter{Introduction}\label{intro}}

\hypertarget{disclaimer}{%
\section{Disclaimer}\label{disclaimer}}

This is a report of how I found it working on the Bernie campaign, mixed with opinions of mine about what happened. I went into 2020 as a Warren and Sanders fan, and so remain. I don't like the Democratic Party. I would rather them to the GOP, but I would rather an unpolished turd to the GOP. This piece is reflective of this point of view. I welcome comment from other points of view (though I rarely get that; friends and even family members are for the most part very unwilling to engage in political discussion here, for fear of I don't know what).

\hypertarget{how-i-ended-up-working-for-bernie}{%
\section{How I ended up working for Bernie}\label{how-i-ended-up-working-for-bernie}}

\hypertarget{section}{%
\subsection{2016}\label{section}}

The Brexit referendum of 2016 frightened me. I desperately wanted Britain to remain in the EU. I voted and hoped other people wouldn't be \emph{stupid} and vote the other way. In other words, I did almost nothing other than feeling clever, followed by feeling disappointed. I had seen other people out campaigning for remain, but somehow I felt that that was a job for \emph{those} people. I still don't understand why. Maybe I didn't care as much as I said I did when I had a pint in my hand, or maybe I was just plain \emph{stupid}. Anyway, I hate it that I did nothing.

\hypertarget{section-1}{%
\subsection{2019}\label{section-1}}

I met an American girl in England in 2015. I married her in 2018 and in 2019 I moved to Oakland, California with her. As Christmas approached, the Democratic Party primary was taking shape. I was particularly excited about Elizabeth Warren and Bernie Sanders. Paul Krugman explained this infatuation well in a January 2020 interview with Ezra Klein, saying ``Something went wrong in the US to lead us to where we are now; only Bernie Sanders and Elizabeth Warren have a coherent theory of what that was.'' This is perhaps unfair on people like Andrew Yang (I cannot understand why he was never taken seriously but Mayor Pete was), but it captures the general idea.

So yes, you could say I'm on the left. I want the government to provide healthcare without individual charges (call that \emph{Medicare for all} if you want to). I want taxes to be raised on people as lucky as myself (I'm an overpaid tech worker) or luckier to pay for things for people who don't have enough. I want to stop fossil fuel exploration. I want women and minorities to get equal pay for equal work. And so on.

Towards the end of 2019, Elizabeth Warren was rising in the polls. She almost caught Joe Biden; it was an exciting time. Being a front-runner coming from the left, she faced an enormous amount of scrutiny. The New York Times, Washington Post, MSNBC, CNN and others pummeled her on the affordability of her \emph{Medicare for All} plan, which was originally just an endorsement of Sanders' plan. The pressure told. She wriggled around and came up with her own plan, slightly more \emph{moderate} than the original. Warren the flip flopper was kicked a few more times when she was down and that was the end of her. She had a notable moment left in her---roasting Mike Bloomberg alive on live TV---but she never had a shot again.

Meanwhile Bernie Sanders had a heart attack and in a quite incredible feat of loyalty from his supporters, managed to brush it off and carry on with his campaign. He seemed to avoid the intense criticism on medicare for all that Warren sustained. You're allowed to believe in that if you're not going to win.

In fall (that's autumn for you Europeans) of 2019, I had tried a few times to get involved in the Warren campaign, but they were slow in getting back to me; they didn't really seem to be up and running in California. When I tried to get involved with the Bernie campaign in early 2020, it was quickly evident that this was already a well-oiled machine in my locality. In I went.

\hypertarget{section-2}{%
\subsection{2020}\label{section-2}}

I'm in, I'm working for Bernie (call me a bro if it makes you feel good). My day job at the time was contract work at Facebook, so I was villain by day, hero at night (at least in my own head). The work at Facebook was so fantastically undemanding that I had a lot of time and energy to devote to Bernie work. I never did Bernie stuff at work, but I could often leave early. The Oakland campaign office was on my block and I was in there more nights than not in the month before Super Tuesday (which includes the California primary).

\hypertarget{bernie-supporters}{%
\chapter{Bernie supporters}\label{bernie-supporters}}

\hypertarget{different-demographics-with-different-concerns}{%
\section{Different demographics with different concerns}\label{different-demographics-with-different-concerns}}

I met lots of wonderful Bernie supporters on the campaign. Most people had one particular issue that they cared about. I think it's interesting for a moment to dwell on who cared about what (of course I'll have to generalise and in reality these groups overlap and I only met a handful of people from these groups).

There were a lot of teachers and nurses, who cared about schools and healthcare. They also cared about their own wages, as they should; their wages under-value their work. These people were lifelong union members, many of them union organisers. Bernie often mentioned teachers and nurses in his stump speeches and teachers and nurses seemed to feel that Bernie really loved and valued them like no other candidate. Being teachers and nurses, the vast majority were women.

There were a hell of a lot of young people. They fear climate change and environmental destruction. I'll put myself into that category. They are also very worried about being able to afford a house; if I had aspirations of owning a home near the San Francisco Bay Area, I'd be worried about that too. As far as I could tell, all of the young people campaigning for Bernie were university educated. I was a bit disappointed by this. It would've been more interesting to have a mix and strategically, this didn't bode well for us. Curiously, many of these young people had never voted before (even though they had been of voting age in previous elections), mostly because they hadn't had a candidate that was far left enough to turn them out. I thought this did bode well for us; turning out new voters was key to the strategy.

There were a lot of old people (ages 60 and above). They were very angry about American foreign policy. Iraq was the last straw for them. The Obama administration's continued support for Saudi Arabia in their war with Yemen was last plus one. For most of them, their bitterness started with the Vietnam war. The no longer trusted what their government or mainstream media or government tell them about foreign policy. For them, Iraq was confirmation of what they already knew. They absolutely hated anyone who voted for that war, Joe Biden and Hillary Clinton being the most notable. These people also hated the US mainstream media. I remember one gentleman recalling that only the \emph{Long Island Newsday} had reported on the US invasion of Panama in 1989. Back then he was still very bitter about Vietnam (he hadn't fought there but he was part of the anti-war protest movement at the time). This media blackout of the Panama invasion in '89 confirmed to him that the US media had not learned to hold the US government to account. He now reads \emph{The Guardian} to get his US news!

The black people that I met were most worried about police brutality against them and the disproportionate number of black people in prison. They hated Mike Bloomberg for \emph{stop and frisk}. I was surprised that the black people I met never raised other issues of systemic racism, for example the pay gap. For them, the immediate threat was police violence; all else was secondary. Chief among secondary concerns was the price of housing. There was a lesson for me there. I feel like I learned a lot more by talking to a few dozen ordinary people than by reading a thousand articles. Before George Floyd's murder, for every article I came across discussing police violence against black people, there were ten that were designed to subtly inform me that I myself am probably subconsciously racist and should be aware of that (and I read them all like a fool). The news should be flooded with stories of unnecessary violence against people of colour and the moral imperative to stop it. We've had a decade of this unconscious bias stuff now. Sure, unconscious biases exist. But the solutions for the people on the ground exist right now and could be implemented right now. Many were in Bernie's policies. If we're to de-unconscious bias our way out of this, it'll take several generations. To wait that long is unconscionable.

\hypertarget{the-work}{%
\chapter{The work}\label{the-work}}

\hypertarget{canvassing}{%
\section{Canvassing}\label{canvassing}}

The most important thing one did on the campaign was canvassing: knocking on doors and talking to people about why Bernie might be right for them. I live in downtown Oakland and there was a high density of volunteers here, so on the weekend we would carpool to places with relatively few volunteers (mostly less-affluent suburbs of Oakland). This was my favourite thing to do, because it really got me out of my bubble (the overpaid tech worker bubble). I'll share some stories later to illustrate why this was so enjoyable.

The canvassing training was very good (I thought). We were told to do as much listening as possible at the door. Let people tell you what they care about and then try to explain why a Bernie presidency aligned with their goals. One needed only a broad overview of Bernie's policy to be competent at the doors. Very few people were interested in the minutiae of the policies, and if they were, you weren't going to change their mind with a two-minute conversation. Most people would mention the price of rent or the fact that they'd once lost their healthcare when they lost their job (this has happened to a terrible number of people) and so a short explanation of Bernie's public housing or \emph{Medicare for all} plans sufficed. Really, you were trying to impart to people that you gave a damn about their struggles, that you wanted them to be helped out.

You never wanted to be in the position of vomiting policies at people. This basically never worked. You always wanted them to be telling you what mattered to them. A common technique was to tell a story of something that had impacted you in the past. My go to was that when I came to the US, being unemployed at first I enrolled in Medicare but I could never get confirmation that I was actually covered. After a couple of months I got a job and bought private insurance. I never really found out when my Medicare coverage began, although I know that at some stage it did. This very boring story actually resonated with a lot of people because so many had had a similar experience. If after telling such a story, you still couldn't trigger people to start talking after the personal story, you were in trouble. At this point, you had no hope other than to tell them about your favorite policies in the hope that one of them would light their eyes up. This never happened.

Most people didn't answer the door. In most of these cases, there probably wasn't anyone home, but sometimes there clearly was. I didn't mind people not answering the door, that's up them. However, in future when people knock on mine, I will definitely answer, knowing the feeling of knocking a dozen doors in a row with no answer. When people did answer, it was almost always a pleasant experience. People could sometimes be nasty over the phone, but almost never in person.

The response to Bernie as a candidate was overwhelmingly positive. Even people who were going to vote for someone else seemed to like Bernie. There was an obvious generational divide. Young people loved Bernie (it still amazes me thinking about the level of enthusiasm that young voters had for this very old man). Middle aged people were on the fence about him, leaning towards Mayor Pete (incidentally I really dislike Mayor Pete because he doesn't seem to stand for anything in particular) for the most part. Older people liked Biden. In particular, older black people really respected Biden for being Obama's VP. They liked Bernie but were firmly with Joe. It was nice to hear these people say how much it meant to their community to see a black man in the White House. There was also an interesting divide among women. Almost all white women that I spoke to were for Warren, whereas black, Asian and Latina women were all about Bernie. I tried to convince these white women that Warren was toast (she was at this stage) and that therefore they should go to Bernie, and told them that I myself had switched from the Warren campaign for this reason (this was a small lie: I \emph{signed up} to be a Warren volunteer twice and they never got back to me and a few weeks later I joined the Bernie campaign). This never worked. There were definitely a lot of women out there who desperately wanted a woman president and were voting accordingly; this is not something that I surmised, the women told me this themselves. I always made sure to ask these women ``Why not Klobuchar'' and they always said it wasn't close between Warren and Klobuchar. I never met a Klobuchar supporter. It was always cordial when I met a Warren supporter, there was often a sense that we were on the same team. I never met a Warren supporter than I didn't like.

The best chance you had at convincing people to vote for your candidate i.e.~gaining a vote was in encounters with habitual voters who were not politically engaged. There were a lot of these people out there: people who believed forcefully in the importance of casting a vote, but seemed relatively unconcerned with who they might end up voting \emph{for}. As I alluded to above, the key with these people was to try to stay friendly while figuring out what they cared about. If you could do that, it was inevitable that you could explain how Bernie planned to address that issue of theirs. This was something that you got better at with practice. The first few doors you knock, you're a babbling mess but before long you get comfortable with what works for you. Mostly, people who answer their door in the first place are friendly. Sometimes they're too busy to talk to you and they tell you so, so you walk away without protest. Still, that person has now had a polite interaction with a Bernie representative. It counts for something.

The other kind of vote gaining that you could do was to find people who already favoured Bernie and simply remind them about when and how to vote. It was amazing how many people didn't know how to go about voting. For example, people who registered as having no party preference would not be able by default to vote in the democratic primary but they could request a ballot with which to do so. Helping people with this amounted to a lot of votes. On polling day, I was out canvassing and reminding people to vote right up until voting closed. As voting closed, just for the hell of it I tried to find the local polling station. It was there aright but it was in a small room at the back of a labyrinthine community centre and assisted living facility. There were no signposts and a lot of people wandering around wondering where exactly to vote. Every so often people who had already found where to go would emerge and send a new flock towards the polling station. There were also a lot of people arriving there right at the deadline. If you were in the line at the time of the deadline, you could vote. Some did arrive too late. Having spend the last two hours wandering around in the cold and dark entreating people to go and vote before the deadline came near, that was a bitter pill to swallow.

I must mention the pain of canvassing people who are politically disengaged. Once someone has told you they have ``no interest'' in voting, there is nothing you can do. You are a self-identified political operative and as such they tune you out. You go through what you were trained to do. ``Do you like other people making decisions for you?'' It is hopeless. I liked may of the disinterested non-voters that I encountered. They were polite enough to stand there and pretend to listen to my routine, and when it was over, they gave a polite goodbye. A key part of the campaign strategy was to turn out this type of person. ``Turn out people who never voted before!'' In the abstract, it made sense. Bernie was an unconventional candidate and as such he probably needed to plot an unconventional path to victory. In the moment though, this strategy was soul destroying. This kind of interaction was far more common in poor neighborhoods, who needed Bernie the most. On the other hand, though, who am I to tell people like this what they need?

I don't want to be too fatalistic about what I am saying here. I think appeals to non-voters should be happening in any functioning democracy. But voting is a cultural thing: it's something that people tend to do if their friends and family also do so. To change non-voters into voters, we need to change the culture around non-voters. This is a difficult and lengthy process, but attempts can be made. For example, people with a university education are more likely to vote. Removing barriers of entry into university education would surely then be a good attempt at increasing voter turnout. (Of course this has the undesired effect of biasing the electorate further towards the university educated, which is a huge caveat.)

\hypertarget{phone-banking}{%
\section{Phone banking}\label{phone-banking}}

Phone banking is just calling people. There were two types of phone banking: calling voters and calling volunteers.

For calling voters, there was an online portal called the ``Bernie Dialer.'' At any given time it would connect you with voters in key strategic parts of the country (usually a state with an upcoming primary). It was interesting to note how different it was to call different parts of the country. If Bernie was popular in that state, it was a nice experience and if not it was rough. California, Washington, Nevada and Texas were nice to call (yes Bernie was surprisingly popular in Texas). Florida was particularly horrible. When people feel like you've bothered them at home to promote someone that they've made up their mind is bad, they often make sure to get an insult in as they hang up the phone. On the whole though, my experiences talking to American strangers were positive. I'll share some specific stories later.

Calling volunteers was mostly about turning people out to canvas. For example, if there was a canvas scheduled in San Lorenzo on a given Saturday, someone like me would obtain a list of people who had in the past signed up as Bernie volunteers and lived local to the canvas area, and call every one of them and ask them to show up. This was always a much more pleasant experience, because everyone you called was a Bernie supporter.

\hypertarget{stories}{%
\chapter{Stories}\label{stories}}

I'll now share some stories about the people I met and the experiences I had.

\hypertarget{the-trump-supporter}{%
\section{The Trump supporter}\label{the-trump-supporter}}

I was out canvassing with a girl called Ali and we knocked on the door of a Trump supporter. We introduced ourselves and asked this woman of (I guess) 60 years how she was feeling about the upcoming election. ``I support Donald Trump'' she said, matter of factly, but not at all aggressively. This was a first. I was feeling all at sea in this situation but Ali was quick to inquire as to why this lady intended to vote for Trump.

The woman made a somewhat difficult to decipher point, the crux of which was that she was a business owner and therefore something something. Ali was again sharp and told the lady that her father was a small business owner and had difficulty affording his employees' healthcare, particularly when times were tough for the business. This was gold, the lady immediately empathised with Ali's father's predicament. I had pulled myself together by this point and began to help Ali. I told the lady that in my view, it should not be down to a business to provide healthcare to its employees, because the employees will lose it when they lose their job and because it puts an unnecessary financial and administrative burden on businesses. She seemed particularly delighted that I mentioned the administrative burden; evidently she had had a lot of trouble sorting this out for her business. ``So I guess you think the government should provide the healthcare?'' she said. I knew I was getting into deep water with this Trump supporter now, but luckily the three of us had already built up a good rapport at this stage.

At this time her daughter (who appeared to be in her late 20s) was also listening in from upstairs, leaning her head over the banister at a most peculiar angle as she spoke and listened. She was in a dressing gown and evidently felt under-dressed to come down and speak to us at the door. Instead she held her head at this angle that the geometry of the house necessitated. I said that indeed the government should provide the healthcare. ``Isn't that socialism?'' she said, with an approving angled nod from the daughter. I said that I believed that some government-led schemes were good, for example small business grants and \emph{social} security. To my great surprise and relief, she was entirely satisfied with this answer, evidently having a very favourable view of social security.

At this point the mother and daughter began to open up to us some more. They told us that the lady's husband was a veteran and that he really wasn't getting the healthcare he needed from the Veteran's Administration (VA). Ali sympathised very well and we both imparted that we thought this a disgrace. It was a good opportunity for us to call out the GOP for being too quick to send soldiers to war and yet reluctant to take good care of them when they return. Ali said that the \emph{government} should ensure that the VA has all of the money it needs to take care of veterans.

The inevitable then happened. We were asked how we would pay for all of this healthcare. I said that healthcare would be paid for through taxation. I said that I would happily pay more taxes so that this veteran could have healthcare, even though I myself hadn't needed the doctor in years. The lady was evidently taken aback by this, but the daughter was by now nodding vociferously, at the same angle. ``You think this is how it should work? You think we should pay more taxes?'' Ali assured her that we would be delighted to pay more taxes if it meant that more people got the healthcare that they needed. The lady found this truly fascinating, as did I. I dreaded having to mention the T word but on the other hand it is at the heart of what I believe in: higher taxes and better services for everyone. It was wonderful to see these beliefs gain legitimacy in this lady's mind.

This was not a fairy tale. We did not manage to convince this woman to change her voter registration and vote for Bernie in the primary. We did, however, convince her that we cared about the problems that she faced with regard to her business and her husband, the veteran. We'll never know to what extent we changed her mind about healthcare provision but I'm pretty sure that she believed that we---a couple of Bernie Sanders supporters---held these views out of good intentions.

It also strikes me that when she opened the door and introduced herself as a Trump supporter, in that moment, her mind was more open than mine. I had already painted an unflattering mental caricature of her. She, was just happy to see us.

\hypertarget{the-man-who-spoke-mandarin}{%
\section{The man who spoke Mandarin}\label{the-man-who-spoke-mandarin}}

We always had a list of houses that we were to visit. Most of the time you knocked and got no answer, which was fine. However, I noticed that it was always worth trying to talk to people who were in the street (walking their dog, doing their garden, etc). People would rarely flat out ignore you in this context.

One such time there was a man of Asian descent out tending his lawn. This seemed like a good call to me, given the condition of the lawn. I approached him and I could tell from a distance that he was immensely shy, so I started my Bernie spiel from a distance in the friendliest way I knew how, whilst slowly closing in on him (poor fellow). Quickly, it became apparent that he didn't have the slightest clue what I was saying, but that he would prefer it if I went away. Using my profiling skills, I fished out a Mandarin Bernie leaflet and continued to approach him, hunched over so as to appear as nonthreatening as possible, all the while holding the leaflet outstretched. He gently took the leaflet and I retreated quickly so as to restore his comfort, saying ``thank you, thank you.'' His eyes lit up when he saw the leaflet; he clearly understood it and that was a pleasant surprise to him. I turned to walk away but he made a happy noise. I turned back around and he was holding the leaflet out towards me, pointing frantically at Bernie's face. ``Like this man! Like this man!'' he said. We shared a very nice smile and I raised my fist into the air to emphasise my delight. That made my day.

\hypertarget{the-girl-with-the-nervous-dog}{%
\section{The girl with the nervous dog}\label{the-girl-with-the-nervous-dog}}

One Saturday I was out canvassing for Bernie with the Democratic Socialists of America (DSA). DSA was not officially affiliated with the Bernie campaign, they had just endorsed him and had the common goal of getting him elected. The DSA people were mostly very young and particularly disenfranchised with the political system. They hated the democrats for being an economically right-wing party. I am largely on board with this, my canonical example being Bill Clinton's 1999 declaration that ``the Glass--Steagall law is no longer appropriate.'' No bona-fide left-wing party thinks like that. Allow me to list another couple of Democrat's that supported Clinton's repeal of that law: Nancy Pelosi, Joe Biden.

Anyway (forgive my temporary lapse into rant mode), I was paired with a girl (let's call her Sally) who had driven all the way from Napa to Oakland to do this canvass (it's a long way so good hustle). She was carrying a small dog in her arms that was visibly nervous; perhaps he didn't like the smell of socialists.

One of the first houses we got to was home to a Warren supporter. It was a mother who was clearly being kept busy by her two toddlers. When she opened the door, her cat escaped. It was kind of getting away down the street and I asked the woman if I should go and get it; she said that I should. Sally was unconcerned, but by this time I had done enough canvassing to know that if we were complicit in the disappearance of the woman's cat, that would be her abiding memory of the visit of these Bernie supporters. With great difficulty (I'm more of a dog person) I managed to shoo the cat back up the street towards the house. Meanwhile, Sally was tearing into this young mother: ``Did you know that Warren voted to increase Trump's war budget?'' Sally had this kind of \emph{gotcha} approach to canvassing whereby she tried to make people feel bad for voting for anyone but Bernie. This is a very tempting strategy, particularly when you don't like the person at the door, but it is \emph{never} effective. We had learned this in the canvassing training, but Sally couldn't help herself. As we left, I told the woman that I liked Warren too. This did not impress her and Sally gave me a look of disdain. The woman did not display any gratitude for the return of her cat.

Down the street we came upon a barbecue outside a church, which was attended by dozens of black people. I approached rather gingerly (it looked like fun and I though the last thing they needed was some white guy coming in and telling them all how to vote). One of them noticed me and was extremely welcoming. When I said I didn't want to spoil any fun, he told me he'd be there all day so it was no big thing to talk to us for 10 minutes. We told him all about Bernie and he was interested and engaged. He said he liked Bernie but that he was thinking of voting for Biden. We spoke for a while and built up a good rapport (I spoke to a hell of a lot of people during the campaign and noticed that black people seemed on average to be much more comfortable talking to strangers). In the end he took a Bernie leaflet and said excitedly that he'd put it on his fridge, but I still thought he would vote for Biden. As we went to leave, he went to pet the nervous dog. The little thing was growling and bearing its teeth, and the man persisted with getting slowly but fearlessly closer to a pat on the dog's head. The inevitable happened. The dog bit him. His finger was bleeding. ``Oh no, what a fucking scene,'' I thought. The guy looked down at his finger and then kindly back towards the dog. ``Is it cos of the color?'', he said to the dog, and then he laughed a really hearty laugh, evidently pleased with his joke. ``Is your finger alright?'', I inquired. He showed me his finger and shot me a smile to let me know that my question was absurd. The man was not \emph{at all} bothered. As we left, he was showing the leaflet to other people at the barbecue, evidently pleased with his interaction with us. You had to admire him. A minute later, Sally turned to me and said ``I knew I should've given the dog some cannabis oil to calm him down before we left,'' as if that were the most obvious thing in the world. I declined to comment in a way that made it known that this was not the most obvious thing in the world to me.

\hypertarget{another-sally-gotcha}{%
\section{Another Sally gotcha}\label{another-sally-gotcha}}

Soon after, a door was opened by a black man of about 60. He said he was voting for Biden. ``He was Obama's vice president,'' he said. It was kind of nice to see this man well up with pride and satisfaction at the thought of a black president. ``But Biden voted to increase Trump's war budget!'' said Sally in a classic gotcha. I should've pointed out that Biden has not held office while Trump was president, but that didn't occur to me at the time. ``Well, he shouldn't have done that. Bernie huh, I'll vote for the guy,'' said the gentleman in the doorway. He then politely ended the conversation and shut the door. There is no chance that that man did not vote for Biden.

\hypertarget{sometimes-youre-not-so-hot}{%
\section{Sometimes you're not so hot}\label{sometimes-youre-not-so-hot}}

One of the doors I knocked on was opened by a young man of about 20. It was always good to see a young face (young people are much more likely to like Bernie). I gave him my spiel and he asked me an open-ended ``what about Bloomberg'' question, telling me that he'd seen a lot of ads for Mikey B (people were surprisingly open about their own susceptibility to TV ads). ``Oh yes'' I thought to myself, ``fish in a barrel.'' I had been canvassing for a while now and had gotten the hang of it. It's difficult to explain what happened. I mumbled some incoherent stuff about Bloomberg having no idea what is wrong in the US and therefore being unable to fix it, mentioning the phrase ``wealth tax'' a few times. The key words are \emph{mumble} and \emph{incoherent}. The young man was perplexed and bored by the time I was finished. I gave him a leaflet. The saving grace was that even though I seemed like an idiot, I think I at least came across as a nice idiot. Nevertheless, I walked away disgusted at a missed opportunity. Not many people were opening their doors that day, so that annoyed me for a while.

\hypertarget{the-lady-from-the-philippines}{%
\section{The lady from the Philippines}\label{the-lady-from-the-philippines}}

Walking around a neighborhood, I came upon a group of six Mexicans having their dinner on their front lawn. Of course, I bothered them. ``No, we are Mexicans'' was the response. ``Are you US citizens?'' I inquired. ``Yes, we can vote, but one Mexican won't make any difference,'' said the man who had by now emerged as the group's spokesman. ``Six Mexicans might,'' I said, clutching at straws. ``Pfff'' was the reaction to that assertion. These people were generally sympathetic to the Bernie message, but didn't feel that their votes mattered. Again, it was almost impossible to turn non-voters into voters. I left these nice Mexicans in peace.

Soon after, a lady from the Philippines opened the door. She was a staunch Bernie supporter and we spoke for a while. She told us (among many other things) that her husband was Mexican. She said he was newly engaged in politics, but also now a strong Bernie supporter. I told her the story of the Mexican group down the block. I don't think they really listen to someone like me (not that they should)," I told her. ``Yes you're right,'' she said, ``they will never listen to \emph{you}.'' ``They might listen to your husband,'' I said, in a clearly leading way. ``Yes, yes, I will send him around to talk to them,'' she said gleefully. I was delighted with this. When you could recruit a volunteer like this, you knew you were picking up several votes at once. As I left, I went to shake her hand (which I did habitually in pre-COVID times). ``Oh no sorry I can't do that, but I can do this,'' she said, prepping her hand for a gentle fist-pump. I guess it was for religious reasons that she couldn't shake my hand, but I'll never know. I felt bad for putting her in this uncomfortable position, but she let on that she didn't mind at all. We exchanged the most delightful fist-pump and I was on my way. What a pleasure it was to talk to her.

\hypertarget{the-warren-supporter-walking-the-dog}{%
\section{The Warren supporter walking the dog}\label{the-warren-supporter-walking-the-dog}}

Another person on the street walking the dog; I always stopped them. This woman was interesting: quite friendly and very open about her thoughts. She was staunchly progressive and evidently keen on Warren. We talked about policy for a while and eventually it was clear that policy-wise, either Bernie or Warren would have done her. ``So why Warren?'' I asked, ``she's a good candidate in my view, but she's toast. Look at how she fared in Iowa, New Hampshire and Nevada.'' ``You know'', she said, ``you're making sense. The thing is, I'm in my seventies, I probably have less than 20 years left and I just want to see a woman president before I die.'' I didn't really know what to say to that, so I said that I hoped very much that she would live to see that and that it would happen sooner rather than later.

I continued on my round and soon enough we intersected with her and her dog again. ``I never said thank you,'' she said. ``I want to thank you for being out her and doing this, it's really great that you're out here. I hope it goes well and keep up the good work.'' Her eager gratitude was heartwarming, particularly coming from the supporter of another candidate.

My interaction with this woman was representative of my experience canvassing. Canvassing is not an adversarial experience. The vast majority of people seem quite pleased to talk to a friendly stranger. Small or even large political differences didn't change that (in my experience).

\hypertarget{the-black-mother-and-daughter-and-mlk}{%
\section{The Black mother and daughter and MLK}\label{the-black-mother-and-daughter-and-mlk}}

I was out canvassing with DSA again with Andrew, a local union organiser. He was much better than me at the whole thing, being more used to working on all sorts of campaigns. We came to a house where a black mother and daughter were getting out of their car after a grocery run. We introduced ourselves as working for Bernie with DSA. The daughter was eager to get inside to avoid being bothered by us; the mother was more curious. The daughter told us that she thought us to be doing a disservice to the black community to self-identify as \emph{socialists} because that word was used to smear MLK. I was all at sea here. I was not about to contend that we were a force for good on racial issues. Andrew (unlike myself) was a real MLK scholar and managed to engage this girl in a polite conversation about him. He didn't manage to talk her around on the point of the use of the word \emph{socialist} but he managed to display his admiration for MLK clearly enough to get us out of that particular bind. Meanwhile, the mother told us that the biggest issue for her was the deficit. That was the only time I heard that issue raised and neither Andrew nor I really fielded that one particularly well. By the end, she seemed to like us for the fact that she believed we were genuine in wanting everyone to have healthcare for everyone, but she wasn't convinced to vote for Bernie.

\hypertarget{betty-the-japanese-woman}{%
\section{Betty the Japanese woman}\label{betty-the-japanese-woman}}

I did group phone banking at a Japanese-American woman's house in Berkeley every Tuesday. Betty was her name and she was around 4ft 10in and a grandmother to a child that would wander in and out (her daughter lived next door). She would serve very nice tea and every time she would have a new array of equally delicious Japanese desserts to go with it. Most of the other phone bankers were at least twice my age, so I could be useful there technology-wise (phone banking required a laptop and a phone and there was a knack to syncing them up to get going).

I looked forward to it every week. Betty would always give me a hug on my way out, which I cherished (it was particularly cute because she was so small). She was a staunch Bernie supporter. My laptop had a Warren sticker on it which Naomi (my wife) got when she signed us up to campaign for Liz. Betty gravely assured me that that wouldn't do at all for a Bernie supporter and she promptly produced a bigger Bernie sticker and pasted it over the Warren sticker. Much better.

The others who attended the phone bank were also interesting. One was the girlfriend of Ted Rall, the author of the \href{https://www.penguinrandomhouse.com/books/536904/bernie-by-ted-rall/}{bestselling graphic novel about Bernie Sanders}. Ted had inspired her to volunteer for Bernie. It was Bernie's \emph{honesty} that impressed her the most. You heard this from a lot of people. They admired Bernie for having a \emph{consistent} message for his decades-long career. One time she arrived with a signed copy of the book for Betty, who was absolutely delighted. It was very sweet.

Another was a retired New York journalist who had written articles for many famous and not so famous publications. She was pretty scathing of people who thought that Bernie was deep down a Soviet-style socialist and insisted that once one could get past that ridiculous thought, Bernie was clearly the best candidate. This was illustrative of older people's view of Bernie. If he didn't terrify you, you liked him.

There was a Berkeley student (early 20s) who came from time to time. He was nice and had a lot of enthusiasm but was a bit over-aggressive in his persuasion efforts. Every week he'd tell us a story of how he'd owned a Warren supporter online in the past few days. He was one of the few true \emph{Bernie bro} people that I met on the campaign.

\hypertarget{what-now}{%
\chapter{What Now}\label{what-now}}

\hypertarget{trust-bernie-and-keep-pushing}{%
\section{Trust Bernie and keep pushing}\label{trust-bernie-and-keep-pushing}}

Bernie has endorsed Joe Biden and together their joint policy task forces have come up with some quite exciting plans on climate and the environment, which are issues dear to my heart. On other issues e.g.~healthcare (Biden's \emph{public option} is a very long way from an unconditional guarantee of healthcare to all) and criminal justice reform (I'm not an \emph{abolish the police} guy, but I do believe that there are a ridiculous number of people in prison here and that cash bail is wrong). Bernie's endorsement of Biden means that he thinks life will be better for the worse off if Trump is defeated. I trust him on that and the idea that ``they've reeled Bernie into the establishment'' is hysterical.

Broadly speaking I am \emph{on the left} and what I consider to be \emph{the left} is still in the minority in this country. My vision for how the left gains power and influence is through democratic means. We came close to getting Bernie elected by those means and I think that's a realistic path to success for us in future. Trump and his goons have no qualms about destroying these democratic means by which we might succeed. They would happily block every minority from voting if they could. Then what are we going to do? Of course it doesn't feel good to be supporting Biden right now, but I think those of us on the left need to be very clear-eyed about how we chart the path to where we want to be. I honestly think the best thing for us to do is to help Biden win and then primary challenge him in four years (if he's even running again). There is no hypocrisy in that, no selling out.

\hypertarget{congress}{%
\section{Congress}\label{congress}}

There are congressional races there to be won. Recently, in NY 16th congressional district, Jamaal Bowman unseated 30-year incumbent Eliot Engel, chair of the foreign affairs committee. Bowman was backed by left groups like the Sunrise Movement, The Jewish Vote and Justice Democrats. About 75,000 votes were cast in that election. Bowman's campaign made over 1,200,000 phone calls, more than 2/3 of which were made by the Sunrise Movement alone. Engel was backed by pretty much the whole democratic establishment including Nancy Pelosi, Chuck Shumer, Hillary Clinton, the Congressional Black Caucus and numerous corporate PACs. Bowman, like Bernie, didn't take money from corporate PACs. It was cool to see him cross the line having thrown some money his way myself, especially having done the same with Bernie.

There are other similar victories throughout the country e.g.~Candace Valenzuela, Mondaire Jones and Mike Siegel. There are also notable defeats e.g.~Andrew Romanoff; indeed we still lose more often than we win. But through force of numbers and conviction, the left is beginning to wield power that up until recently it simply didn't have.

\hypertarget{more-on-biden}{%
\chapter{More on Biden}\label{more-on-biden}}

I get the idea of the left refusing to back a \emph{moderate} candidate against a right-wing candidate and I think it has its merits \emph{if} it's part of a plausible longer term strategy for the left to elect people more sympathetic to its cause. In this case, Trump is hell bent on destroying the very democratic means that the left might hope to use to succeed in future. Although Biden is not overly sympathetic to the left, I do think he believes in the democratic process. Also given the obvious fact that the upcoming election will be won by either Biden or Trump, I think it's in the lefts short and long term best interest to make sure Biden wins, even if it's not a great best.

\bibliography{book.bib}

\end{document}
